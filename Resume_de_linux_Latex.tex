\documentclass[11pt,a4peper]{article}
\usepackage{graphicx}

\begin{document}

\title{Linux Résumé}

\section{\textbf{Résumé Cours de linux}}

Linux est un système d’exploitation opensource avec de nombreuse distribution.

\section{\textbf{Commandes}}

Les commandes sont formés de la commande initial suivis d’attribut, il existe 2 types d’attributs :\\
- Attribut de type Linux avec un tiret avant l’attribut\\
- Attribut de type unix avec deux tiret avant l’attribut\\
Pour connaître le fonctionnement des commandes et leurs attributs il y a 2 methodes :\\
- " man commande " pour afficher le manuel de la commande\\
- " commande --help pour afficher une aide de la commande\\

Dans le cas ou on ne connaît pas le nom de la commande qu’on doit utiliser on peut utiliser le man pour nous aider avec l’attribut -k pour rechercher un mot clé.\\
Exemple : man -k user      pour trouver les commandes relatifs au user\\
\\
- whoaim : " Qui suis-je " donne le nom de l’utilisateur\\
- hostname : donne le nom de la machine\\
- date : donne la date actuelle\\
- uname: donne des informations sur le système actuel\\
- passwd : pour changer son mot de passe ou le mot de passe d’un autre utilisateur en rajoutant le     nom de celui-ci après passwd\\
- touch: crée un fichier ou permet la modification de la date d’accès et de modification du fichier\\
- last : liste des utilisateurs qui se sont connecté \\
- history : affiche les commandes utilisées\\
- ls : liste les fichiers et dossiers du répertoire actuel ( -l pour lister, -a pour tout afficher)\\
- hier : description de la hiérarchie du système de fichiers\\
- cp : copier un fichier vers répertoire attribut -r  pour un dossier (cp [SOURCE] [DESTINATION])\\
- cd : permet de se déplacer dans l’arborescence de fichier\\
- mkdir : création d’un dossiers\\
- rmdir : supprime un dossier vide ( rm -rf  pour supprimer un dossier et son contenu)\\
- rm : supprime un fichier\\
- mv : déplace un fichier, peut-être utiliser pour le renommer dans le déplaçant dans le même dossier\\
- find : permet de trouver des fichiers \\

Il existe 3 canaux principaux dans Linux :\\
- STDIN : c’est l’entrée standard (généralement le clavier)\\
- STDOUT : c’est la sortie standard (généralement l’écran)\\
- STDERR : c’est le canal d’erreur (généralement vers un fichier)\\

\includegraphics[scale=0.65]{C:/Users/calin/Documents/Latex/Photo 1.png}\\

On pourrait utiliser " > " a la suite d’une commande pour envoyer le résultat dans un ficher\\
Exemple : history > history.txt\\

En Linux le pipe permet de rediriger la sortie d’une commandes dans l’entrée d’une seconde afin que la deuxième commande effectue un traitement sur le résultat de la première \\
Exemple : on veut utiliser ls mais ne voir que les fichiers .jpeg   "  ls -la | grep .jpeg "\\

\section{L’arborescence du système de fichier}

\includegraphics[scale=0.7]{C:/Users/calin/Documents/Latex/Photo 2.png} \\
Cette structure peut sensiblement varier en fonction des distributions , mais un tronc commun est communément admis c’est le " FHS : file hierarchy standard \\

Chaque arborescence de fichier en Linux prend toujours naissance avec le " root directory " ou " / " \\

Depuis le " / " l’arborescence se dessine autour de dossiers fondamentaux pour le fonctionnement du système \\

Ce système de fichier peut être héberger sur un seul device de stockage\\
- HDD\\
- SSD\\
- Etc\\
Cependant, il est courant et conseillé d’isoler certains dossiers sur des devices différents afin de repartir la charge de travail de ceux-ci, par exemple /home qui sera très utilisé pourrait être mis sur un HDD réservé a /home. La commande mount permet cette manipulation\\

\includegraphics[scale=0.5]{C:/Users/calin/Documents/Latex/Photo 3.png} 

\section{Les wildcarts}

Le SHELL est capable d’interpréter des symboles de remplacements dans les commandes.\\

* : remplace plusieurs caractère inconnus. \\

? : remplace un caractère inconnu. \\

[0-9] : remplace un caractère par un des caractères du range défini dans cette exemple entre 0 et 9\\

\section{Chemin absolu et relatif}

Un chemin absolu est un chemin qui commence à la racine du système de fichier, commençant par\\
" / "\\
Un chemin relatif est un chemin qui commence à la position actuelle dans le système de fichier. \\

\section{Hard-Link et le Symbolic-Link}

Le Hard-Link : est un nom qui référence un " inode " qui lui même référence un bloc sur le périphérique de stockage. \\

Le symbolic-Link : est un nom qui référence un Hard-link, équivalent d’un raccourcis sur windows\\

\includegraphics[scale=0.52]{C:/Users/calin/Documents/Latex/Photo 4.png} 

ln [SOURCE] [LINK-NAME] : permet de créer un HARD-LINK \\

ln -s [SOURCE] [LINK-NAME] : permet de créer un Symbolic-Link \\

\section{\textbf{Travail des fichiers textes}}

\section{VIM}

La commande " VIM " : Edition d’un fichier texte.\\
Vim (lien du fichier) / ex : vim /home/admin/documents/texte\\
Mode commande (ESC) : sauvegarder, quitter, rechercher …\\
Mode insert (i) : éditer le texte\\
Mode visual(v) : Selection dans le texte\\

Plusieurs options en mode " VIM " :\\

-En mode VISUAL :\\
D (delete la sélection)\\
Y (copie la sélection)\\
P (colle la sélection)\\

-En mode INSERT :\\
Utilisation comme un éditeur de texte sans l’option copier – coller.\\


-En mode COMMANDE : !!! vous pouvez les enchainer !!!\\
-w (sauvegarde votre fichier)\\
-q (quitte le fichier)\\
-! ( force l’action)\\
-wq ! (sauvegarde et quitte en forçant l’action)\\
-u (undo)\\
-Ctrl-r (redo)\\
-300 (mène à la ligne 300)\\
-/test (recherche le mot test)\\

\section{LESS}

La commande " less " : organise le texte d’une page pour le " SHELL "\\
less source / ex : less /var/log/dnf.log\\
-La navigation ,de page en page, se fait avec page-up et page-down\\
-Recherche avec :/[mot recherché]\\
-" Q " pour quitter\\

\section{CAT et TAC}

cat [source] (présente le fichier) ex : cat /home/admin/documents/catfile\\
tac [source] (présente le fichier en commençant par la fin \\
 Ex : tac /home/admin/documents/catfile
 
 \section{HEAD et TAIL}
 
head -n[nbres de lignes] [sources] (présente les premières lignes du fichier.\\
Ex : head -n5 /home/admin/documents/catfile (présente les 5 premières lignes du fichier)\\
tail -n[nbres de lignes] [sources] (présente les dernières lignes du fichier.\\
Ex : tail -n5 /home/admin/documents/catfile (présente les 5 dernières lignes du fichier)\\
PS : un -f affiche les 10dernières lignes et les actualises à chaque changement.\\

\section{GREP}

grep permet de trouver une expression dans un texte contenu dans un fichier.\\
Avec un pipe " | " on peut préciser la recherche dans un dossier particulier. [cat source | grep expression]\\
Exemple : cat /home/admin/documents/catfile | grep ‘one’  (recherche du mot one)

\section{AWK , SORT et TR}

"awk est une commande qui permet de découper un texte en fonction de ses délimiteurs."\\
awk -F [délimiteur] ‘{print (dolars)[numéro colonne]}’ [source]\\
Ex : awk -F : ‘{print (dolars)1}’ home/admin/documents/catfile\\
sort trie le texte par ordre alphabétique ou numérique.\\
 sort [source] (si on veut par ordre numérique (-n))\\
Ex : sort (-n) home/admin/documents/catfile\\
tr permet de réaliser une traduction de certains caractères du texte en d’autres \\
tr [caractère initial] [caractère par lequel il doit changer]\\
Ex : echo ‘ping pong’ |tr [p] [t] - résultat ‘ting tong’\\

\section{Se connecter à un serveur}
\section{Root et local user}

-Le root est un super-admin, c’est l’utilisateur le plus puissant de la machine\\
Les ‘local user’ sont des utilisateurs tout à fait normaux.\\
Deux façons de passer root : 	-sudo -i\\
								-sudo su –\\
Il est possible de travailler en root mais il est déconseillé !\\
Utilisation de la commande " su " : elle sert à se connecter au root et à n’importe quelle use. / ex : sudo su – (root) ; sudo su Joseline (user : Joseline)\\
L’utilisation d’une commande en tant qu’admin se fait en rajoutant sudo avant la commande.\\

\section{Connection a une machine linux à distance}

Ssh [username]@[adresse-ip] (connection sécurisé)\\
Telnet [username]@[adresse-ip] (connection sécurisé)

\section{La gestion des utilisateurs}
\section{Création utilisateur}

Afin de créer un user :\\
Useradd [option disponible avec un -h] [login/nom user]\\
Ex : useradd -m josseline (création de l’user josseline avec un home directory)\\
Pour attribuer un groupe à l’user :	-g groupe principal\\
									-G groupe secondaire\\
									
\section{Création de groupes}

Afin de créer un groupe :\\
Groupadd [option liste -h] [nom du groupe]\\
Ex : groupadd -g 2500 secrétariat (création du groupe secrétariat avec comme id 2500) 

\section{Modification de user / groupe existant}

Modification user :\\
Usermod [option] [login/nom utilisateur]\\
Usermod -aG secretariat josseline (assigné le groupe secretariat a josseline sans écrasé les autres(-a))\\
Quelque option sympa :	\\
-" -L " : Lock du compte \\	
-" -U " : Unlock du compte \\
-" -e " : Fait expirer le password\\
-" -u " : Modifie l’UID associé au login \\
-" -I " : Modifie le nom du login\\
Supprimé un user /groupe :\\
Userdel -r [login/username]\\
Groupdel (-f) [nom du groupe]\\
Modification d’un groupe : un -h met les listes des options\\
Groupmod -g [new id group] changé l’id du groupe par exemple\\
Dossiers défaut utilisateur : etc/default/useradd et etc/login.defs\\
Useradd se trouve le groupe par default, expiration si défini, squelette du home, spooling mail.\\
Login.defs se trouve le chemin spool, droit default, validité pswd, création groupe au nom du user et création du home.\\
Gestion passwords\\
passwd [option dispo avec -h] [login/username]\\
chage [options][login/username] Permet la même chose que Passwd avec une meilleure lisibilité.\\

/etc/passwd: contient la liste des utilisateurs et est en accès non restreint.\\
Il contient 7champs : Login : password : UID : GID : comment : homedir : shell
/etc/group: contient la définition des groupes et la liste des utilisateurs qui en font partie.\\
Il contient 4champs : Group: password : GID : users (si des users tierces font partie du groupe)\\
/etc/shadow: Le fichier shadow accompagne le fichier /etc/passwd et c’est là que sont stockés, entre autres, les passwords cryptés des utilisateurs ainsi que les informations relatives à leur validité.\\
- Il contient 9 champs : \\
- Champ 1 : login \\
- Champ 2 : le password crypté : le (dolars)xx(dolars) indique le type de cryptage.\\
- Champ 3 : le nombre de jours depuis le 01/01/1970 depuis le dernier changement de password. \\
- Champ 4 : le nombre de jours avant lesquels le password ne peut pas être changé. \\
- Champ 5 : le nombre de jours après lesquels le password doit être changé. \\
- Champ 6 : le nombre de jours avant l’expiration du password durant lesquels le user doit être prévenu. \\
- Champ 7 : le nombre de jours après l’expiration du password après lesquels le compte user est désactivé. \\
- Champ 8 : le nombre de jours depuis le 01/01/1970 à partir du moment où le compte user a été désactivé. \\
- Champ 9 : est le champ réservé. \\
Le cryptage des password sera le plus souvent en " (dolars)6(dolars) : encryptage SHA-512 ".\\

\section{Système de permisssion}
\section{Permissions standards}

-chmod [UGO ou OCTAL][PATH]  (rwx / 777)\\
Ceci permet de gérer les permissions des différent folders, pour la lecture, l'écriture et l'éxecution.

\includegraphics[scale=0.9]{C:/Users/calin/Documents/Latex/Photo 5.png} 

\includegraphics[scale=0.9]{C:/Users/calin/Documents/Latex/Photo 6.png} 

\section{Permissions spéciales}

(UGO) -chmod [Spécial][UGO][PATH]: chmod u+s folder1 \\
(OCTAL) -chmod [Spécial][OCTAL][PATH]: chmod 1777 folder1
[Spécial]\\
u+s / 4: Active le SUID\\
g+s / 2: Active le SGID\\
+t / 1: Active le STICKY BIT

\includegraphics[scale=0.46]{C:/Users/calin/Documents/Latex/Photo 7.png}

\section{Access list}

-setfacl -Rm [Permissions Modification] [File/FOLDER PATH]

Normal : appliquée sur les files déjà existant.\\
-setfacl -Rm g:bob:rx /home/admin/Documents/folder1/\\
Default : appliquée sur les files qui seront nouvellement créé.\\
-setfacl -Rm d:g:bob:rx /home/admin/Documents/folder1/ \\
Pour vérifier l'ACL, utiliser "getfacl [File/FOLDER PATH]"

\section{Gestion des Quotas}

Les quotas permettent de limiter l'utilisation de l'espace disque de chaque groupe/utilisateur.

Deux types de limites:
- Block: Limite sur la taille du dossier\\
- Inodes: Limite sur le nombre de fichiers (dossiers)\\

Blocage soft: Limite que l'utilisateur/groupe peut dépasser pendant un certain temps.\\
Blocage hard: Limite que l'utilisateur/groupe ne pourra jamais dépasser.

\section{Configuration réseau "run time"}

Il y a deux types de configuration.
- La configuration " run time " conçue pour monitorer les paramètres et pour effectuer des tests.
- La configuration persistante faites pour fournir à votre machine un accès permanent et fiable au réseau.

Pour le mode Run time, nous allons utiliser -ip [OPTIONS][OBJETS]
Pour voir les interfaces réseau disponibles sur votre système: -ip link
Pour voir les interfaces réseaux disponibles et leur adresse: -ip address show

\includegraphics[scale=0.5]{C:/Users/calin/Documents/Latex/Photo 8.png} 

"man ip" pour obtenir plus d'informations.

Pour le mode de configuration persistante, nous allons utiliser "nmcli" avec des auto complétion.
Man nmcli pour plus d'informations.

Pour vérifier la configuration du réseau effectuée, utilisez 
-ping [IP ADDRESS/HOSTNAME]
-dig [HOSTNAME]

\section{Le Firewalling sous Linux}

Le processus de firewalling en Linux est géré par Netfilter et est directement intégré au Kernel.\\
Netfilter travaille par filtre des " process informations " et gère les capacités " input " " output " " forward " dans le kernel.\\
Afin de manipuler netfilter, il faut passer par la commande -iptables\\
En iptables, on travail sur les échanges qui on lieu entre:\\
- INPUT : pour les paquets entrants\\
- OUTPUT : pour les paquets sortant\\
- FORWARD : dans le cadre d’un routeur, pour les paquets transférés\\
-iptables -A [INPUT/OUTPUT/FORWARD] [-i/-o] [INTERFACE] -p [udp/tcp/icmp [--dport/sport [n°]] -j [LOG/ACCEPT/DROP/REJECT]

\includegraphics[scale=0.5]{C:/Users/calin/Documents/Latex/Photo 9.png} 

Les Well-known ports

\includegraphics[scale=0.5]{C:/Users/calin/Documents/Latex/Photo 10.png} 

Il existe aussi le firewall-cmd\\
firewall-cmd est une surcouche à iptables et permet de manière plus simple gérer les " INPUT " et les " OUTPUT " sur votre système.\\
Comme pour iptables, il est important de connaître le status des autorisations trafiques sur votre système.\\
-firewall-cmd --list-all\\
La configuration de firewall-cmd est basée sur l’autorisation de services.\\
L’ensemble de ces services sont visibles grâce à la commande suivante :\\
-firewall-cmd --get-services\\

Pour maintenant accorder l’utilisation d’un service et l’autoriser à communique à travers le firewall\\
-firewall-cmd --add-service ssh (version runtime)\\
-firewall-cmd --add-service ssh --permanent (version persistante)\\
Pour maintenant supprimer l’utilisation d’un service et l’empêcher de communique à travers le " firewall "\\
-firewall-cmd --remove-service ssh --permanent (version persistante)\\
Après avoir imposé une nouvelle règle, il est essentiel de reload le service " firewalld "\\
-firewall-cmd –reloa

\end{document}
